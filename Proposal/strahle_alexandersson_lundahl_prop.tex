
\documentclass{article}
\usepackage[english]{babel}
\usepackage[utf8]{inputenc}
\usepackage{amsmath} 
\usepackage{fancyhdr}
\usepackage{lastpage}
\usepackage{enumerate}
\usepackage{lineno}
\usepackage{lmodern}
\usepackage{caption}
\usepackage[T1]{fontenc}
\usepackage{microtype}
\usepackage{systeme}
\usepackage{amsmath,amssymb,amsthm,mathrsfs,latexsym,tikz,url}
\usepackage{epigraph,graphicx}
\usepackage{listings}
\usepackage{listingsutf8}
\usepackage{color}
\usepackage{float}

\DeclareGraphicsExtensions{.png,.pdf}
\definecolor{dkgreen}{rgb}{0,0.6,0}
\definecolor{gray}{rgb}{0.5,0.5,0.5}
\definecolor{mauve}{rgb}{0.58,0,0.82}

\lstset{frame=tb,
  language=Python,
  aboveskip=3mm,
  belowskip=3mm,
  showstringspaces=false,
  columns=flexible,
  basicstyle={\small\ttfamily},
  numbers=none,
  numberstyle=\tiny\color{gray},
  keywordstyle=\color{blue},
  commentstyle=\color{dkgreen},
  stringstyle=\color{mauve},
  breaklines=true,
  breakatwhitespace=true,
  tabsize=4
}


\usepackage{hyperref}
\hypersetup{
	colorlinks=true,
	linkcolor=blue,
	filecolor=magenta,      
	urlcolor=cyan,
}
\urlstyle{same}

\setlength{\parindent}{0.0cm}
\setlength{\parskip}{0.1cm}
\setlength{\voffset}{-1in}

\begin{document}

\title{DD2424: Project Proposal \\
%The impact of Fourier, and other, transforms in Deep Learning \\
The mechanisms, powers and limitations of some Data Augmentation techniques}

\author{Anton Stråhle, Jan Alexandersson \& Fredrika Lundahl}
\maketitle 

\section{Project Description}

%In this project we aim to observe the mechanism and impact of various data augmentation techniques and transforms on the accuracy of a convolutional neural network. Whilst we want to implement and observe the effects of many data augmentation techniques, as well as combinations of several of them, we specifically want to observe how the Fourier transform impacts the accuracy and when. Another technique that we want to focus a bit more on is the usage of \href{https://arxiv.org/pdf/1710.09412.pdf}{Mixup} and how different weight distributions affect this method.

%***Alternativt stycke:
In this project we aim to observe the mechanism and impact of various data augmentation techniques and transforms on the accuracy of a convolutional neural network. Whilst we want to implement and observe the effects of many data augmentation techniques, as well as combinations of several of them, we specifically want to observe how different types of data augmentations impacts the accuracy and when. For example we want to try simple techniques as rotating and flipping, the usage of \href{https://arxiv.org/pdf/1710.09412.pdf}{Mixup} and how different weight distributions affect this method as well as investigating whether Fourier transformations can be of use. 
We may also try, if time left, only performing mixup on the images and let majority in the mixup decide the label.

\medskip

In all, the focus is not entirely on maximizing the accuracy on a specific data set through a very complex CNN (although some decently performing base architecture will be used) but rather the observation of how different data augmentation techniques can affect the accuracy and when they do. It would of course be terrific to achieve a very high accuracy but this is not the main goal of the project.

\section{Data}

Our plan is to use a \href{https://www.kaggle.com/gpiosenka/100-bird-species}{Bird Species Dataset}, containing about 25 000 labeled images of 180 different bird species. 

\section{Deep Learning Packages \& Implementation}

We will work in Python and use TensorFlow. We will hopefully be able to implement all the code necessary ourselves.

\section{Experiments}

Initially we want to create a CNN that without augmentation achieves a decent accuracy on the data set in question and leaves some room for improvements. After that we want to experiment how certain data augmentation techniques, and the combination of these, affect this accuracy and in which settings they are powerful. To create different settings we intend to use different subsets of the dataset with different number of training examples and different subsets of bird species.

Some specific techniques that we want to try are

\begin{enumerate}[(i)]
 \item Fourier transform
 \item Mixup
 \item Color magnification 
 \item Translations and Rotations
\end{enumerate}

If we have enough time we might also try the techniques on a vastly different data set.

\section{Measurement of Success}

As our project is more of an observational study where we only wish to examine the effect of and gain 
understanding about certain transforms we do not target high accuracy. 
Instead our success can be measured in how many data augmentation techniques, and 
combinations of these, we can draw conclusions about and see how much it can increase the accuracy.

\section{Knowledge \& Skills}

All of us want to establish a solid base in TensorFlow and get and understanding on the effects of different data augmentations techniques, i.e. when and how to use them.
All members in our group would also want to improve our programming skills in python since we mostly use R as master students in mathematical statistics at Stockholm University. 

\section{Grade}

We are aiming for $\text{B}\geq$.

\end{document}

© 2020 GitHub, Inc.
Terms
Privacy
Security
Status
Help
Contact GitHub
Pricing
API
Training
Blog
About
Found 2 out of 2 items